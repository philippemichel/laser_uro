\documentclass[a4paper,fontsize=10pt,french]{scrartcl}
%
\usepackage{geometry}
\usepackage{graphicx}
\usepackage{rotating}
\usepackage{amsmath}
\usepackage{array}
\usepackage{fontspec}
\usepackage{microtype}
\usepackage[output-decimal-marker={,}]{siunitx}
\usepackage[french]{varioref}
\usepackage{tikz}
\usepackage{textcomp}
\usepackage{tabularx}
\usepackage{booktabs}
\usepackage{longtable}
\usepackage{babel}
\usepackage{hyperref}
%
\hypersetup{%
colorlinks=true,
pdftitle={},
pdfauthor={Philippe MICHEL},
pdfkeywords={},
unicode
}
%
  %
\setmainfont[Ligatures=TeX,
BoldFont = MinionPro-Bold,
ItalicFont = MinionPro-It]
   {MinionPro-Regular}
   \setsansfont[Ligatures=TeX]
{MyriadPro-Regular}

\title{Laser en urologie}
\usepackage{etoolbox}
\subtitle{Évenements indésirables}
\author{Philippe MICHEL}
\date{05 novembre 2021 - V3szz}

\begin{document}
\maketitle

\tableofcontents


\section{Description des données}\label{description-des-donnuxe9es}


Pour avoir un groupe plus homogène \& des comparaisons qui ont un sens
on ne garde que les interventions où le laser  est parfois  utilisé à
savoir :

\begin{itemize}
\item
  Calcul urinaire
\item
  Hyperplasie de la prostate
\item
  Tumeur bénigne de la prostate
\item
  Dysplasie de la prostate
\item
  Affection de la prostate
\item
  Colique néphrétique
\item
  Incontinence urinaire
\item
  Dysurie
\item
  Congestion et hémorragie prostatiques
\item
  Autres calculs des voies urinaires inférieures
\end{itemize}

La base de données comprend alors 363 cas pour 48 dont 149 cas de laser.
le nombre de variables devient donc important \& il est hors de question
de faire des tests au hasard sur toutes les variables.

Les données des variables \textsf{Evénement'} \&  \textsf{CCAM} ont été nettoyées \&
regroupées. Seuls les dix intitulés les plus fréquents seront pris en
compte.

\subsection{Données démographiques}\label{donnuxe9es-duxe9mographiques}

\begin{table}

\caption{\label{tab:tabcomp}Données démographiques}
\centering
\begin{tabular}[t]{l|l|l|l}
 \toprule
  & autre & laser & p\\
\midrule
\textbf{Sexe} &  &  & p = 0.055\\
 
féminin & 80/264 (30.3\,\%) & 20/99 (20.2\,\%) & \\
 
masculin & 184/264 (69.7\,\%) & 79/99 (79.8\,\%) & \\
\midrule
\textbf{Âge} & 64.8 ± 14.8 & 64.7 ± 13.9 & p = 0.956\\
\midrule
\textbf{Taille} & 169 ± 14.4 & 172 ± 6.29 & p = 0.057\\
\midrule
\textbf{Poids} & 75.5 ± 13.3 & 78.5 ± 13.8 & p = 0.056\\
\midrule
\textbf{IMC} &  &  & p = 0.721\\
 
Underweight & 3/261 (1.1\,\%) & 0/99 (0\,\%) & \\
 
Normal weight & 102/261 (39.1\,\%) & 40/99 (40.4\,\%) & \\
 
Overweight & 123/261 (47.1\,\%) & 45/99 (45.5\,\%) & \\
 
Obese & 33/261 (12.6\,\%) & 14/99 (14.1\,\%) & \\
\midrule
\textbf{Grosssesse} &  &  & p = 0.566\\
 
non & 75/80 (93.8\,\%) & 20/20 (100\,\%) & \\
 
oui & 5/80 (6.2\,\%) & 0/20 (0\,\%) & \\
\midrule
\textbf{Score ASA} &  &  & p = 0.737\\
 
1 & 59/117 (50.4\,\%) & 21/47 (44.7\,\%) & \\
 
2 & 33/117 (28.2\,\%) & 16/47 (34\,\%) & \\
 
3 & 25/117 (21.4\,\%) & 10/47 (21.3\,\%) & \\
\midrule
\textbf{Complexité clinique} &  &  & p = 0.008\\
 
Non complexe & 174/247 (70.4\,\%) & 37/67 (55.2\,\%) & \\
 
Plutôt non complexe & 47/247 (19\,\%) & 13/67 (19.4\,\%) & \\
 
Plutôt complexe & 26/247 (10.5\,\%) & 17/67 (25.4\,\%) & \\
 
Très complexe & 0/247 (0\,\%) & 0/67 (0\,\%) & \\
 \bottomrule
\end{tabular}
\end{table}

Les patients ``Laser'' ont une \emph{complexité clinique} plus
importante. Pas d'autre différence notable.

\begin{figure}
  \includegraphics[width = \linewidth]{laser_uro_files/figure-latex/pyr-1} 
  %\includegraphics[width=\linewidth]{laser_uro_files/figure-latex/pyr-1}
  %\caption{Pyramide des âges}
  %\label{figpyr}
\end{figure}

\subsection{Diagnostic}\label{diagnostic}

\subsubsection{CIM 10}\label{cim-10}

\includegraphics[width = \linewidth]{laser_uro_files/figure-latex/cim-1}

\subsubsection{CCAM}\label{ccam}

\begin{table}

\caption{\label{tab:pyr}Libellés CCAM les plus fréquents}
\centering
\begin{tabularx}{\textwidth}[t]{X|r}
  \toprule
 & n\\
autre & 80\\
Résection d'une hypertrophie de la prostate, par endoscopie & 49\\
Résection d'une hypertrophie de la prostate, par urétrocystoscopie & 39\\ 
Fragmentation intrarénale de calcul avec ondes de choc ou laser [Lithotritie intrarénale], par urétéronéphroscopie & 22\\
Soutènement vésical par bandelette synthétique infra-urétrale, par voie transvaginale et par voie transobturatrice & 22\\
Ablation et/ou fragmentation de calcul de l'uretère lombal, par urétéroscopie rétrograde & 21\\
Pose d'une endoprothèse urétérale, par endoscopie rétrograde & 18\\
Ablation et/ou fragmentation de calcul de l'uretère pelvien, par urétéroscopie rétrograde & 17\\
Fragmentation intrarénale de calcul caliciel inférieur avec ondes de choc ou laser [Lithotritie intrarénale], par urétéronéphroscopie & 16\\
Lithotritie extracorporelle du rein, avec guidage radiologique & 16\\
Biopsie de la prostate, par voie transrectale avec guidage échographique & 10\\
  \bottomrule
\end{tabularx}
\end{table}

\subsection{Autres données}\label{autres-donnuxe9es}



\centering
\begin{longtable}[t]{l|l|l|l}
  \caption{\label{tab:tabcomp}Facteurs liés à l'intervention}\\
 \toprule
  & autre & laser & p\\
  \midrule
  \endfirsthead
 \toprule
  & autre & laser & p\\
  \midrule
  \endhead
  \bottomrule
  \endfoot
    \bottomrule
  \endlastfoot
\textbf{cause.immediate.principale} &  &  & p = 0.016\\
CLINIQUE & 28/264 (10.6\,\%) & 7/99 (7.1\,\%) & \\
GESTE TECHNIQUE & 40/264 (15.2\,\%) & 14/99 (14.1\,\%) & \\
INFORMATION & 92/264 (34.8\,\%) & 25/99 (25.3\,\%) & \\
MATERIEL ET STERILISATION & 55/264 (20.8\,\%) & 38/99 (38.4\,\%) & \\
MEDICAMENT & 49/264 (18.6\,\%) & 15/99 (15.2\,\%) & \\
\midrule
\textbf{situation.a.risque} &  &  & p = 0.998\\
non & 127/262 (48.5\,\%) & 48/99 (48.5\,\%) & \\
oui & 135/262 (51.5\,\%) & 51/99 (51.5\,\%) & \\
\midrule
\textbf{but.acte.medical} &  &  & p = 0.098\\
Dépistage & 1/264 (0.4\,\%) & 0/99 (0\,\%) & \\
Diagnostic & 11/264 (4.2\,\%) & 0/99 (0\,\%) & \\
Thérapeutique & 252/264 (95.5\,\%) & 99/99 (100\,\%) & \\
\midrule
\textbf{produit} &  &  & p = 0.141\\
Dispositif médical & 16/100 (16\,\%) & 10/30 (33.3\,\%) & \\
Dispositif médical / Autre produit de santé & 6/100 (6\,\%) & 2/30 (6.7\,\%) & \\
 
Dispositif médical implantable & 27/100 (27\,\%) & 4/30 (13.3\,\%) & \\
 
Médicament & 51/100 (51\,\%) & 14/30 (46.7\,\%) & \\
 
Médicament / Dispositif médical & 0/100 (0\,\%) & 0/30 (0\,\%) & \\
 
Médicament / Dispositif médical implantable & 0/100 (0\,\%) & 0/30 (0\,\%) & \\
\midrule
\textbf{rayonnements.ionisants} &  &  & p = 0.167\\
 
ne sais pas & 6/264 (2.3\,\%) & 0/99 (0\,\%) & \\
 
non & 250/264 (94.7\,\%) & 98/99 (99\,\%) & \\
 
oui & 8/264 (3\,\%) & 1/99 (1\,\%) & \\
\midrule
\textbf{localisation.survenue} &  &  & p = 0.314\\
 
Autre, préciser & 16/264 (6.1\,\%) & 9/99 (9.1\,\%) & \\
 
Bloc opératoire & 145/264 (54.9\,\%) & 58/99 (58.6\,\%) & \\
 
ne sais pas & 1/264 (0.4\,\%) & 1/99 (1\,\%) & \\
 
Salle de cathétérisme, d'endoscopie, d'exploration fonctionnelle\dots & 0/264 (0\,\%) & 0/99 (0\,\%) & \\
 
Salle de surveillance post-interventionnelle & 9/264 (3.4\,\%) & 0/99 (0\,\%) & \\
 
Service de radiologie & 1/264 (0.4\,\%) & 0/99 (0\,\%) & \\
 
Service des urgences & 6/264 (2.3\,\%) & 0/99 (0\,\%) & \\
 
Unité de consultation, salle de pansement & 7/264 (2.7\,\%) & 1/99 (1\,\%) & \\
 
Unité de réanimation, de soins intensifs & 1/264 (0.4\,\%) & 0/99 (0\,\%) & \\
 
Unité Hospitalisation & 78/264 (29.5\,\%) & 30/99 (30.3\,\%) & \\
\midrule
\textbf{periode.vulnerable} &  &  & p = 0.012\\
 
non & 210/264 (79.5\,\%) & 89/98 (90.8\,\%) & \\
 
oui & 54/264 (20.5\,\%) & 9/98 (9.2\,\%) & \\
\midrule
\textbf{periode} &  &  & p = 0.803\\
 
Autre & 13/54 (24.1\,\%) & 4/9 (44.4\,\%) & \\
 
Heure de changement d'équipe & 7/54 (13\,\%) & 0/9 (0\,\%) & \\
 
Heure de changement d'équipe / Autre & 3/54 (5.6\,\%) & 0/9 (0\,\%) & \\
 
Jour férié & 1/54 (1.9\,\%) & 0/9 (0\,\%) & \\
 
Jour férié / Autre & 1/54 (1.9\,\%) & 0/9 (0\,\%) & \\
 
Jour férié / Week-end / Autre & 0/54 (0\,\%) & 0/9 (0\,\%) & \\
 
Nuit & 10/54 (18.5\,\%) & 1/9 (11.1\,\%) & \\
 
Nuit / Autre & 1/54 (1.9\,\%) & 1/9 (11.1\,\%) & \\
 
Nuit / Heure de changement d'équipe & 1/54 (1.9\,\%) & 0/9 (0\,\%) & \\
 
Nuit / Heure de changement d'équipe / Autre & 0/54 (0\,\%) & 0/9 (0\,\%) & \\
 
Nuit / Week-end & 1/54 (1.9\,\%) & 0/9 (0\,\%) & \\
 
Nuit / Week-end / Autre & 1/54 (1.9\,\%) & 0/9 (0\,\%) & \\
 
Week-end & 13/54 (24.1\,\%) & 3/9 (33.3\,\%) & \\
 
Week-end / Autre & 2/54 (3.7\,\%) & 0/9 (0\,\%) & \\
\midrule
\textbf{degre.urgence.pac} &  &  & p = 0.106\\
 
Non urgent & 174/261 (66.7\,\%) & 78/98 (79.6\,\%) & \\
 
Urgence différable de quelques heures & 32/261 (12.3\,\%) & 9/98 (9.2\,\%) & \\
 
Urgence différable de quelques jours & 22/261 (8.4\,\%) & 5/98 (5.1\,\%) & \\
 
Urgence immédiate & 33/261 (12.6\,\%) & 6/98 (6.1\,\%) & \\
\midrule
\textbf{pac.programmee} &  &  & p = 0.107\\
 
non & 51/264 (19.3\,\%) & 12/99 (12.1\,\%) & \\
 
oui & 213/264 (80.7\,\%) & 87/99 (87.9\,\%) & \\
\midrule
\textbf{infection.nosocomiale} &  &  & p = 0.646\\
 
non & 250/261 (95.8\,\%) & 93/99 (93.9\,\%) & \\
 
oui & 11/261 (4.2\,\%) & 6/99 (6.1\,\%) & \\
\midrule
\textbf{niveau.gravite} &  &  & p = 0.331\\
 
1 - mineur & 189/264 (71.6\,\%) & 63/99 (63.6\,\%) & \\
 
2 - significatif & 55/264 (20.8\,\%) & 27/99 (27.3\,\%) & \\
 
3 - majeur & 12/264 (4.5\,\%) & 4/99 (4\,\%) & \\
 
4 - grave à  critique & 6/264 (2.3\,\%) & 5/99 (5.1\,\%) & \\
 
5 - catastrophique & 2/264 (0.8\,\%) & 0/99 (0\,\%) & \\
\end{longtable}


Il y a plus d'incidents avec le Laser en non urgent comparé aux autres
méthodes mais est-ce que le Laser n'est utilisé préférentiellement dans
des indications programmées ? Même réflexion pour les autres items.

\subsection{Causes suspectées}\label{causes-suspectuxe9es}

\begin{table}

\caption{\label{tab:tabcomp}Causes identifiées}
\centering
\begin{tabularx}{\linewidth}[t]{X|l|l|l}
  \toprule
  & autre & laser & p\\
\midrule
\textbf{identification.causes.immediates} &  &  & p = 0.671\\
 
non & 28/264 (10.6\,\%) & 9/99 (9.1\,\%) & \\
 
oui & 236/264 (89.4\,\%) & 90/99 (90.9\,\%) & \\
\midrule
\textbf{causes.liees.patient} &  &  & p = 0.362\\
 
ne sais pas & 4/264 (1.5\,\%) & 0/99 (0\,\%) & \\
 
non & 186/264 (70.5\,\%) & 67/99 (67.7\,\%) & \\
 
oui & 74/264 (28\,\%) & 32/99 (32.3\,\%) & \\
\midrule
\textbf{causes.liees.taches} &  &  & p = 0.026\\
 
ne sais pas & 12/264 (4.5\,\%) & 2/99 (2\,\%) & \\
 
non & 133/264 (50.4\,\%) & 37/99 (37.4\,\%) & \\
 
oui & 119/264 (45.1\,\%) & 60/99 (60.6\,\%) & \\
\midrule
\textbf{causes.liees.soignant} &  &  & p = 0.749\\
 
ne sais pas & 15/264 (5.7\,\%) & 4/99 (4\,\%) & \\
 
non & 143/264 (54.2\,\%) & 57/99 (57.6\,\%) & \\
 
oui & 106/264 (40.2\,\%) & 38/99 (38.4\,\%) & \\
\midrule
\textbf{cause.individu} &  &  & p = 0.533\\
 
Facteurs de stress physique ou psychologique & 12/67 (17.9\,\%) & 3/25 (12\,\%) & \\
 
Qualifications, compétences & 49/67 (73.1\,\%) & 21/25 (84\,\%) & \\
 
Qualifications, compétences / Facteurs de stress physique ou psychologique & 6/67 (9\,\%) & 1/25 (4\,\%) & \\
\midrule
\textbf{causes.liees.equipe} &  &  & p = 0.873\\
 
ne sais pas & 4/264 (1.5\,\%) & 2/99 (2\,\%) & \\
 
non & 116/264 (43.9\,\%) & 41/99 (41.4\,\%) & \\
 
oui & 144/264 (54.5\,\%) & 56/99 (56.6\,\%) & \\
\midrule
\textbf{causes.liees.environnement.travail} &  &  & p = 0.89\\
 
ne sais pas & 8/264 (3\,\%) & 4/99 (4\,\%) & \\
 
non & 184/264 (69.7\,\%) & 68/99 (68.7\,\%) & \\
 
oui & 72/264 (27.3\,\%) & 27/99 (27.3\,\%) & \\
\midrule
\textbf{causes.liees.organisation} &  &  & p = 0.02\\
 
ne sais pas & 5/264 (1.9\,\%) & 2/99 (2\,\%) & \\
 
non & 194/264 (73.5\,\%) & 58/99 (58.6\,\%) & \\
 
oui & 65/264 (24.6\,\%) & 39/99 (39.4\,\%) & \\
\midrule
\textbf{causes.liees.institutionnel} &  &  & p = 0.482\\
 
ne sais pas & 5/264 (1.9\,\%) & 3/99 (3\,\%) & \\
 
non & 244/264 (92.4\,\%) & 93/99 (93.9\,\%) & \\
 
oui & 15/264 (5.7\,\%) & 3/99 (3\,\%) & \\
  \bottomrule
\end{tabularx}
\end{table}

Les causes liées à la tâche \& à l'organisatrion sont plus souvent
notées pour le laser.

\includegraphics[width=1\linewidth]{laser_uro_files/figure-latex/graphcause-1}

La seule cause d'incident où une différence est retrouvée est le
matériel.

\subsection{Après l'évènemement}\label{apruxe8s-luxe9vuxe8nemement}

\begin{table}

\caption{\label{tab:tabcomp}Conséquences de l'incident}
\centering
\begin{tabular}[t]{l|l|l|l}
  \toprule
  & autre & laser & p\\
\midrule
\textbf{identifile.barrieres.efficasses} &  &  & p = 0.478\\
 
non & 83/264 (31.4\,\%) & 35/99 (35.4\,\%) & \\
 
oui & 181/264 (68.6\,\%) & 64/99 (64.6\,\%) & \\
\midrule
\textbf{qualifieriez.vous} &  &  & p = 0.158\\
 
évitable & 231/264 (87.5\,\%) & 77/99 (77.8\,\%) & \\
 
probablement évitable & 22/264 (8.3\,\%) & 16/99 (16.2\,\%) & \\
 
probablement inévitable & 5/264 (1.9\,\%) & 3/99 (3\,\%) & \\
 
inévitable & 2/264 (0.8\,\%) & 2/99 (2\,\%) & \\
 
ne sais pas & 4/264 (1.5\,\%) & 1/99 (1\,\%) & \\
\midrule
\textbf{existe.recommandation} &  &  & p = 0.451\\
 
ne sais pas & 73/264 (27.7\,\%) & 28/99 (28.3\,\%) & \\
 
non & 75/264 (28.4\,\%) & 34/99 (34.3\,\%) & \\
 
oui & 116/264 (43.9\,\%) & 37/99 (37.4\,\%) & \\
\midrule
\textbf{au.sein.equipe} &  &  & p = 0.499\\
 
ne sais pas & 2/264 (0.8\,\%) & 0/99 (0\,\%) & \\
 
non & 13/264 (4.9\,\%) & 3/99 (3\,\%) & \\
 
oui & 249/264 (94.3\,\%) & 96/99 (97\,\%) & \\
\midrule
\textbf{rmm} &  &  & p = 0.279\\
 
ne sais pas & 15/264 (5.7\,\%) & 2/99 (2\,\%) & \\
 
non & 200/264 (75.8\,\%) & 75/99 (75.8\,\%) & \\
 
oui & 49/264 (18.6\,\%) & 22/99 (22.2\,\%) & \\
\midrule
\textbf{sein.etablissement} &  &  & p = 0.668\\
 
ne sais pas & 12/264 (4.5\,\%) & 3/99 (3\,\%) & \\
 
non & 117/264 (44.3\,\%) & 41/99 (41.4\,\%) & \\
 
oui & 135/264 (51.1\,\%) & 55/99 (55.6\,\%) & \\
\midrule
\textbf{al.ars} &  &  & p = 0.283\\
 
ne sais pas & 16/264 (6.1\,\%) & 3/99 (3\,\%) & \\
 
non & 245/264 (92.8\,\%) & 96/99 (97\,\%) & \\
 
oui & 3/264 (1.1\,\%) & 0/99 (0\,\%) & \\
\midrule
\textbf{autres.institutions} &  &  & p = 0.443\\
ne sais pas & 17/264 (6.4\,\%) & 3/99 (3\,\%) & \\
non & 240/264 (90.9\,\%) & 93/99 (93.9\,\%) & \\
oui & 7/264 (2.7\,\%) & 3/99 (3\,\%) & \\
\midrule
\textbf{patient.informe} &  &  & p = 0.262\\
ne sais pas & 7/264 (2.7\,\%) & 0/99 (0\,\%) & \\
non & 78/264 (29.5\,\%) & 30/99 (30.3\,\%) & \\
oui & 179/264 (67.8\,\%) & 69/99 (69.7\,\%) & \\
  \bottomrule
\end{tabular}
\end{table}

Les incidents survenus avec le Laser semblent moins ``évitables''.

\includegraphics[width=1\linewidth]{laser_uro_files/figure-latex/graphap-1}

\section{Matériel \& Laser}\label{matuxe9riel-laser}

Les problèmes liés au matériel semblent plus fréquents avec le Laser;
Tout d'abord regardons si ces problèmes liés avec le matériel peuvent
avoir d'autres causes.

\hypertarget{tableau-comparatif}{%
\subsection{Tableau comparatif}\label{tableau-comparatif}}

\begin{table}

\caption{\label{tab:tabcomp}Causes des incidents liés au matériel}
\centering
\begin{tabular}[t]{l|l|l|l}
  \toprule
  & Autre cause & Matériel et stérilisation & p\\
\midrule
\textbf{age} & 66.6 ± 14 & 59.4 ± 14.8 & p < 0,001\\
\midrule
\textbf{bmi} &  &  & p = 0.763\\
 
Underweight & 3/269 (1.1\,\%) & 0/91 (0\,\%) & \\
 
Normal weight & 105/269 (39\,\%) & 37/91 (40.7\,\%) & \\
 
Overweight & 125/269 (46.5\,\%) & 43/91 (47.3\,\%) & \\
 
Obese & 36/269 (13.4\,\%) & 11/91 (12.1\,\%) & \\
\midrule
\textbf{asa} &  &  & p = 0.003\\
 
1 & 51/121 (42.1\,\%) & 29/43 (67.4\,\%) & \\
 
2 & 37/121 (30.6\,\%) & 12/43 (27.9\,\%) & \\
 
3 & 33/121 (27.3\,\%) & 2/43 (4.7\,\%) & \\
\midrule
\textbf{avant.complexite.clinique} &  &  & p = 0.921\\
 
Non complexe & 159/238 (66.8\,\%) & 52/76 (68.4\,\%) & \\
 
Plutôt non complexe & 45/238 (18.9\,\%) & 15/76 (19.7\,\%) & \\
 
Plutôt complexe & 34/238 (14.3\,\%) & 9/76 (11.8\,\%) & \\
 
Très complexe & 0/238 (0\,\%) & 0/76 (0\,\%) & \\
\midrule
\textbf{but.acte.medical} &  &  & p = 0.37\\
 
Dépistage & 1/270 (0.4\,\%) & 0/93 (0\,\%) & \\
 
Diagnostic & 10/270 (3.7\,\%) & 1/93 (1.1\,\%) & \\
 
Thérapeutique & 259/270 (95.9\,\%) & 92/93 (98.9\,\%) & \\
\midrule
\textbf{periode.vulnerable} &  &  & p = 0.312\\
 
non & 219/269 (81.4\,\%) & 80/93 (86\,\%) & \\
 
oui & 50/269 (18.6\,\%) & 13/93 (14\,\%) & \\
\midrule
\textbf{pac.programmee} &  &  & p = 0.103\\
 
non & 52/270 (19.3\,\%) & 11/93 (11.8\,\%) & \\
 
oui & 218/270 (80.7\,\%) & 82/93 (88.2\,\%) & \\
\midrule
\textbf{situation.a.risque} &  &  & p < 0,001\\
non & 105/270 (38.9\,\%) & 70/91 (76.9\,\%) & \\
oui & 165/270 (61.1\,\%) & 21/91 (23.1\,\%) & \\
\midrule
\textbf{laser} &  &  & p < 0,001\\
autre & 209/270 (77.4\,\%) & 55/93 (59.1\,\%) & \\
laser & 61/270 (22.6\,\%) & 38/93 (40.9\,\% ) & \\
 \bottomrule
\end{tabular}
\end{table}

Les incidents liés au matériel \& stérilisation semblent plus fréquents
chez les patients plus jeunes, moins sévères (score ASA) \& où la
situation à risque n' pas été retenue \& avec un Laser. En clair pour
des interventions complexes, avec du gros matériel, réalisées de jour
sans urgence.

\includegraphics[width=1\linewidth]{laser_uro_files/figure-latex/mat0-1}
Les problèmes rencontrées pendant les interventions avec Laser semblent
différentes des autres interventions (p = p = 0.016).

Les problèmes liés au matériel semblent fréquents lors des interventions
avec Laser. On se concentre sur les types d'interventions où le Laser
est parfois utilisé.

\includegraphics[width=1\linewidth]{laser_uro_files/figure-latex/mat1-1}

\begin{verbatim}
## 
##              tt$laser
## tt$materiel   autre laser Total
##   autre cause   209    61   270
##   matériel       55    38    93
##   Total         264    99   363
## 
## OR =  2.37 
## 95% CI =  1.43, 3.91  
## Chi-squared = 11.64, 1 d.f., P value = 0.001
## Fisher's exact test (2-sided) P value = 0.001
\end{verbatim}

Incident lié au matériel selon l'usage du Laser ou non.

\subsection{Analyse multivariée}\label{analyse-multivariuxe9e}

\includegraphics[width=1\linewidth]{laser_uro_files/figure-latex/matmulti-1}

En analyse multivariée par régression logistique, Les cas d'incidents
liés au matériel \& à la stérilisation se produisent plutôt chez des
patients plus jeunes, moins sévères, en prise en charge programmée \& en
dehors d'une situation à risque mais surtout quand un Laser est utilisé.
pour simplifier, en urgence ou sur des patients sévères on utilise du
matériel plus simple donc moins de problèmes.

\section{Technique}\label{technique}

\textbf{Petit paragraphe à ajouter en fin de ``matériel \& méthode''}

Les données discrètes ont été décrites par leur fréquence exprimée en
pourcentage avec son intervalle de confiance à 95 \% et ont été
comparées par le test exact de Fisher vu la faible effectif. Les
intervalles de confiance n'ont été réalisés qu'après transformation
angulaire. Les données numériques ont été décrites par leur moyenne
(avec son intervalle de confiance à 95 \% calculé par bootstrap) et
l'écart-type. Les données continues ont été comparées par le test de
Student après vérification de l'égalité des variances. Les analyses
multivariées ont été réalisées par régression logistique. tous les items
ayant un p \textless{} 20 \% ont ét utilisés. Une recherche du meilleur
modèle a ensuite été réanisée par stet by step descendant. Les
statistiques ont été réalisées grâce au logiciel R\cite{rstat} avec
en particulier les packages du Tidyverse\cite{tidy} \&
factoMineR\cite{facto}.

\bibliographystyle{plain-fr}
\bibliography{stat}
\end{document}
